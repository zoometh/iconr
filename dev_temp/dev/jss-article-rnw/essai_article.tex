\documentclass[article]{jss}

%% -- LaTeX packages and custom commands ---------------------------------------

%% recommended packages
\usepackage{thumbpdf,lmodern}
\usepackage{float,tikz}
%% another package (only for this demo article)
\usepackage{framed}
%% author pakckage
\usepackage{xcolor,caption}

\graphicspath{ {./imgs/} }

%% new custom commands
\newcommand{\class}[1]{`\code{#1}'}
\newcommand{\fct}[1]{\code{#1()}}


%% -- Article metainformation (author, title, ...) -----------------------------

%% - \author{} with primary affiliation
%% - \Plainauthor{} without affiliations
%% - Separate authors by \And or \AND (in \author) or by comma (in \Plainauthor).
%% - \AND starts a new line, \And does not.
\author{Thomas Huet\\UMR 5140}
%   \And Second Author\\Plus Affiliation}
\Plainauthor{Thomas Huet}

%% - \title{} in title case
%% - \Plaintitle{} without LaTeX markup (if any)
%% - \Shorttitle{} with LaTeX markup (if any), used as running title
\title{Modelling Pre- and Protohistorical Iconographic Compositions. The \proglang{R} package \pkg{decorr}}
\Plaintitle{Modelling Pre- and Protohistorical Iconographic Compositions. The R package 'decorr'}
\Shorttitle{Modelling Pre- and Protohistorical Iconographic Compositions}

%% - \Abstract{} almost as usual
\Abstract{
  By definition, Prehistorical societies are characterised by the absence of a writing system. Writing is one of the most rational semiographical system with a clear distinction between signified and signifier -- specially in alphabetic and binary writings -- and the development of the signified on a horizontal, vertical or boustrophedon axis. Prehistorical times cover more than 99\% of the human livin
}


\section[Introduction: Count data regression in R]{Introduction: Count data regression in \proglang{R}} \label{sec:intro}


\begin{figure}%
    \includegraphics[width=20cm]{"D:/Sites_10/Al Ula FINAL/BD/(verif)/NA_data"} }}%
    \caption{XXX}%
    \label{fig:example}%
\end{figure}


\end{document}
