\documentclass[article]{jss}\usepackage[]{graphicx}\usepackage[]{color}
% maxwidth is the original width if it is less than linewidth
% otherwise use linewidth (to make sure the graphics do not exceed the margin)
\makeatletter
\def\maxwidth{ %
  \ifdim\Gin@nat@width>\linewidth
    \linewidth
  \else
    \Gin@nat@width
  \fi
}
\makeatother

\definecolor{fgcolor}{rgb}{0.345, 0.345, 0.345}
\newcommand{\hlnum}[1]{\textcolor[rgb]{0.686,0.059,0.569}{#1}}%
\newcommand{\hlstr}[1]{\textcolor[rgb]{0.192,0.494,0.8}{#1}}%
\newcommand{\hlcom}[1]{\textcolor[rgb]{0.678,0.584,0.686}{\textit{#1}}}%
\newcommand{\hlopt}[1]{\textcolor[rgb]{0,0,0}{#1}}%
\newcommand{\hlstd}[1]{\textcolor[rgb]{0.345,0.345,0.345}{#1}}%
\newcommand{\hlkwa}[1]{\textcolor[rgb]{0.161,0.373,0.58}{\textbf{#1}}}%
\newcommand{\hlkwb}[1]{\textcolor[rgb]{0.69,0.353,0.396}{#1}}%
\newcommand{\hlkwc}[1]{\textcolor[rgb]{0.333,0.667,0.333}{#1}}%
\newcommand{\hlkwd}[1]{\textcolor[rgb]{0.737,0.353,0.396}{\textbf{#1}}}%
\let\hlipl\hlkwb

\usepackage{framed}
\makeatletter
\newenvironment{kframe}{%
 \def\at@end@of@kframe{}%
 \ifinner\ifhmode%
  \def\at@end@of@kframe{\end{minipage}}%
  \begin{minipage}{\columnwidth}%
 \fi\fi%
 \def\FrameCommand##1{\hskip\@totalleftmargin \hskip-\fboxsep
 \colorbox{shadecolor}{##1}\hskip-\fboxsep
     % There is no \\@totalrightmargin, so:
     \hskip-\linewidth \hskip-\@totalleftmargin \hskip\columnwidth}%
 \MakeFramed {\advance\hsize-\width
   \@totalleftmargin\z@ \linewidth\hsize
   \@setminipage}}%
 {\par\unskip\endMakeFramed%
 \at@end@of@kframe}
\makeatother

\definecolor{shadecolor}{rgb}{.97, .97, .97}
\definecolor{messagecolor}{rgb}{0, 0, 0}
\definecolor{warningcolor}{rgb}{1, 0, 1}
\definecolor{errorcolor}{rgb}{1, 0, 0}
\newenvironment{knitrout}{}{} % an empty environment to be redefined in TeX

\usepackage{alltt}

%% -- LaTeX packages and custom commands ---------------------------------------

%% recommended packages
\usepackage{thumbpdf,lmodern}
\usepackage{float,tikz}
%% another package (only for this demo article)
\usepackage{framed}
%% author pakckage
\usepackage{xcolor,caption, float}

\graphicspath{ {./imgs/} }

%% new custom commands
\newcommand{\class}[1]{`\code{#1}'}
\newcommand{\fct}[1]{\code{#1()}}

% <<preliminaries, echo=FALSE, results=hide>>=


%% -- Article metainformation (author, title, ...) -----------------------------

%% - \author{} with primary affiliation
%% - \Plainauthor{} without affiliations
%% - Separate authors by \And or \AND (in \author) or by comma (in \Plainauthor).
%% - \AND starts a new line, \And does not.
\author{Thomas Huet\\UMR 5140}
%   \And Second Author\\Plus Affiliation}
\Plainauthor{Thomas Huet}

%% - \title{} in title case
%% - \Plaintitle{} without LaTeX markup (if any)
%% - \Shorttitle{} with LaTeX markup (if any), used as running title
\title{Modelling Prehistorical Iconographic Compositions. The \proglang{R} package \pkg{decorr}}
\Plaintitle{Modelling Prehistorical Iconographic Compositions. The R package 'decorr'}
\Shorttitle{Modelling Prehistorical Iconographic Compositions}

%% - \Abstract{} almost as usual
\Abstract{
  By definition, Prehistorical societies are characterised by the absence of a writing system. Prehistorical times cover more than 99\% of the human living. Even if it is being discussed, first symbolic manifestations start around 200,000 BC \citep{dErrico00}. The duration from first symbolic expressions to start of writing represents 97\% of the human living. In illiterate societies, testimonies of symbolic systems mostly come from iconography (ceramic decorations, rock-art, statuary, etc.) and signs are displayed mostlty a discontinuous figures which can have different relationships one with another. An graphical composition can be "read" as a spatial distribution of features having intrinsic values possibily having meaningful relationships one with another depending on their pairwise spatial proximities. 

  To understand meaningful associations of signs, geometric tools, graph analysis and statistical analysis offer great tools to recognize iconographical patterns and to infer collective conventions. We present the \pkg{decorr} \proglang{R} package which ground concepts, methods and tools to analyse ancient graphical systems.
}

\Keywords{Iconography, Prehistory, Graph Theory, Graph Drawing, Spatial Analysis, \proglang{R}}
\Plainkeywords{Iconography, Prehistory, Graph Theory, Graph Drawing, Spatial Analysis, R}

\Address{
  Thomas Huet\\
  CNRS-UMR 5140 \\
  Archeologie des Societes Mediterraneennes\\
  Universite Paul Valery\\
  route de Mende\\
  Montpellier 34199, France\\
  E-mail: \email{thomashuet7@gmail.com}\\
%  URL: \url{https://eeecon.uibk.ac.at/~zeileis/}
}
\IfFileExists{upquote.sty}{\usepackage{upquote}}{}
\begin{document}
% \Sweave2knitr("article_rvTH14_1.Rnw")
\SweaveOpts{concordance=TRUE}

\section[Introduction]{Introduction} \label{sec:intro}

For decades, study of ancient iconography was linked to history of religion because closely linked to symbolism, believes and religions. Since the \textit{New Archaeology} developpement during the 60's \citep{Clarke14}, symbolic expressions start to be studied with the same formal methods (statistics, seriations, distribution maps, etc.) as any another aspect of social organisation: settlement patterns, tools \emph{chaine opératoire}, susbsitence strategies, etc. \citep{Renfrew91}, \citep{LeroiGourhan92}. But unlike many aspects of the material culture -- a flint blade for cutting, a pottery for containing, a house for living --, the function of an iconographic composition cannot be drawn directly from itself. Whether study of ancient iconography had  undergone significative improvements at the site scale -- with GIS, database, paleoclimatic restitutions, etc. -- and at the sign scale with the development of archaeological sciences -- radiocarbon dating, use-wear analysis, elemental analysis, etc. --, these improvement do not necessarly help to understand the semantic content of the iconography.
Semantics or semiotics can be defined as a system of conventional signs organised also in conventional manners.  
Until our days, formal methods to study ancient iconography Semantics, has been mostlty been grounded (explicitly or not) on the prime principle of Saussurian linguistic: the 'linearity of the signifier' \citep{Saussure89}.  
Writing is one of the most rational semiographical system. With a clear distinction between signified and signifier -- specially in alphabetic and binary writings -- and the development of the signified on a horizontal, vertical or boustrophedon axis. 
Let us take the example of the word "\code{art}" which contains three vertices (\code{a}, \code{r}, \code{t}) and two edges (one between \code{a} and \code{r}, the other between \code{r} and \code{t}). In \proglang{R}, these features, concatenated in this order with a \code{paste0()}, is \code{art}, and not \code{rat}

% \SweaveOpts{width=3,height=1}
\begin{figure}[H]
\begin{knitrout}
\definecolor{shadecolor}{rgb}{0.969, 0.969, 0.969}\color{fgcolor}
\includegraphics[width=\maxwidth]{figure/unnamed-chunk-1-1} 

\end{knitrout}
\captionof{figure}[directional links]{concatenate of \code{a}, \code{r} and \code{t} graphical units (GUs) is \code{art}.}
\end{figure}

But, as stated, in Prehistorical the writing system does not exists. Spatial relationships between graphical features, or graphical units (GUs) are not necessarly linear and directed but could most probably be more multi-directional and undirected: the direction of the interactions of pairwise GUs can be in any order.  

\begin{figure}[H]
\begin{knitrout}
\definecolor{shadecolor}{rgb}{0.969, 0.969, 0.969}\color{fgcolor}
\includegraphics[width=\maxwidth]{figure/unnamed-chunk-2-1} 

\end{knitrout}
\captionof{figure}[multidirectional links]{Potential spatial relations between \code{a}, \code{r} and \code{t} GUs.}
\end{figure}

\end{document}
